\documentclass[a4paper,12pt,oneside]{memoir}

% Castellano
\usepackage[spanish,es-tabla]{babel}
\selectlanguage{spanish}
\usepackage[utf8]{inputenc}
\usepackage[T1]{fontenc}
\usepackage{lmodern} % Scalable font
\usepackage{microtype}
\usepackage{placeins}

\RequirePackage{booktabs}
\RequirePackage[table]{xcolor}
\RequirePackage{xtab}
\RequirePackage{multirow}

% Links
\PassOptionsToPackage{hyphens}{url}\usepackage[colorlinks]{hyperref}
\hypersetup{
	allcolors = {red}
}

% Ecuaciones
\usepackage{amsmath}

% Rutas de fichero / paquete
\newcommand{\ruta}[1]{{\sffamily #1}}

% Párrafos
\nonzeroparskip

% Huérfanas y viudas
\widowpenalty100000
\clubpenalty100000

% Imágenes

% Comando para insertar una imagen en un lugar concreto.
% Los parámetros son:
% 1 --> Ruta absoluta/relativa de la figura
% 2 --> Texto a pie de figura
% 3 --> Tamaño en tanto por uno relativo al ancho de página
\usepackage{graphicx}
\newcommand{\imagen}[3]{
	\begin{figure}[!h]
		\centering
		\includegraphics[width=#3\textwidth]{#1}
		\caption{#2}\label{fig:#1}
	\end{figure}
	\FloatBarrier
}

% Comando para insertar una imagen sin posición.
% Los parámetros son:
% 1 --> Ruta absoluta/relativa de la figura
% 2 --> Texto a pie de figura
% 3 --> Tamaño en tanto por uno relativo al ancho de página
\newcommand{\imagenflotante}[3]{
	\begin{figure}
		\centering
		\includegraphics[width=#3\textwidth]{#1}
		\caption{#2}\label{fig:#1}
	\end{figure}
}

% El comando \figura nos permite insertar figuras comodamente, y utilizando
% siempre el mismo formato. Los parametros son:
% 1 --> Porcentaje del ancho de página que ocupará la figura (de 0 a 1)
% 2 --> Fichero de la imagen
% 3 --> Texto a pie de imagen
% 4 --> Etiqueta (label) para referencias
% 5 --> Opciones que queramos pasarle al \includegraphics
% 6 --> Opciones de posicionamiento a pasarle a \begin{figure}
\newcommand{\figuraConPosicion}[6]{%
  \setlength{\anchoFloat}{#1\textwidth}%
  \addtolength{\anchoFloat}{-4\fboxsep}%
  \setlength{\anchoFigura}{\anchoFloat}%
  \begin{figure}[#6]
    \begin{center}%
      \Ovalbox{%
        \begin{minipage}{\anchoFloat}%
          \begin{center}%
            \includegraphics[width=\anchoFigura,#5]{#2}%
            \caption{#3}%
            \label{#4}%
          \end{center}%
        \end{minipage}
      }%
    \end{center}%
  \end{figure}%
}

%
% Comando para incluir imágenes en formato apaisado (sin marco).
\newcommand{\figuraApaisadaSinMarco}[5]{%
  \begin{figure}%
    \begin{center}%
    \includegraphics[angle=90,height=#1\textheight,#5]{#2}%
    \caption{#3}%
    \label{#4}%
    \end{center}%
  \end{figure}%
}
% Para las tablas
\newcommand{\otoprule}{\midrule [\heavyrulewidth]}
%
% Nuevo comando para tablas pequeñas (menos de una página).
\newcommand{\tablaSmall}[5]{%
 \begin{table}
  \begin{center}
   \rowcolors {2}{gray!35}{}
   \begin{tabular}{#2}
    \toprule
    #4
    \otoprule
    #5
    \bottomrule
   \end{tabular}
   \caption{#1}
   \label{tabla:#3}
  \end{center}
 \end{table}
}

%
% Nuevo comando para tablas pequeñas (menos de una página).
\newcommand{\tablaSmallSinColores}[5]{%
 \begin{table}[H]
  \begin{center}
   \begin{tabular}{#2}
    \toprule
    #4
    \otoprule
    #5
    \bottomrule
   \end{tabular}
   \caption{#1}
   \label{tabla:#3}
  \end{center}
 \end{table}
}

\newcommand{\tablaApaisadaSmall}[5]{%
\begin{landscape}
  \begin{table}
   \begin{center}
    \rowcolors {2}{gray!35}{}
    \begin{tabular}{#2}
     \toprule
     #4
     \otoprule
     #5
     \bottomrule
    \end{tabular}
    \caption{#1}
    \label{tabla:#3}
   \end{center}
  \end{table}
\end{landscape}
}

%
% Nuevo comando para tablas grandes con cabecera y filas alternas coloreadas en gris.
\newcommand{\tabla}[6]{%
  \begin{center}
    \tablefirsthead{
      \toprule
      #5
      \otoprule
    }
    \tablehead{
      \multicolumn{#3}{l}{\small\sl continúa desde la página anterior}\\
      \toprule
      #5
      \otoprule
    }
    \tabletail{
      \hline
      \multicolumn{#3}{r}{\small\sl continúa en la página siguiente}\\
    }
    \tablelasttail{
      \hline
    }
    \bottomcaption{#1}
    \rowcolors {2}{gray!35}{}
    \begin{xtabular}{#2}
      #6
      \bottomrule
    \end{xtabular}
    \label{tabla:#4}
  \end{center}
}

%
% Nuevo comando para tablas grandes con cabecera.
\newcommand{\tablaSinColores}[6]{%
  \begin{center}
    \tablefirsthead{
      \toprule
      #5
      \otoprule
    }
    \tablehead{
      \multicolumn{#3}{l}{\small\sl continúa desde la página anterior}\\
      \toprule
      #5
      \otoprule
    }
    \tabletail{
      \hline
      \multicolumn{#3}{r}{\small\sl continúa en la página siguiente}\\
    }
    \tablelasttail{
      \hline
    }
    \bottomcaption{#1}
    \begin{xtabular}{#2}
      #6
      \bottomrule
    \end{xtabular}
    \label{tabla:#4}
  \end{center}
}

%
% Nuevo comando para tablas grandes sin cabecera.
\newcommand{\tablaSinCabecera}[5]{%
  \begin{center}
    \tablefirsthead{
      \toprule
    }
    \tablehead{
      \multicolumn{#3}{l}{\small\sl continúa desde la página anterior}\\
      \hline
    }
    \tabletail{
      \hline
      \multicolumn{#3}{r}{\small\sl continúa en la página siguiente}\\
    }
    \tablelasttail{
      \hline
    }
    \bottomcaption{#1}
  \begin{xtabular}{#2}
    #5
   \bottomrule
  \end{xtabular}
  \label{tabla:#4}
  \end{center}
}



\definecolor{cgoLight}{HTML}{EEEEEE}
\definecolor{cgoExtralight}{HTML}{FFFFFF}

%
% Nuevo comando para tablas grandes sin cabecera.
\newcommand{\tablaSinCabeceraConBandas}[5]{%
  \begin{center}
    \tablefirsthead{
      \toprule
    }
    \tablehead{
      \multicolumn{#3}{l}{\small\sl continúa desde la página anterior}\\
      \hline
    }
    \tabletail{
      \hline
      \multicolumn{#3}{r}{\small\sl continúa en la página siguiente}\\
    }
    \tablelasttail{
      \hline
    }
    \bottomcaption{#1}
    \rowcolors[]{1}{cgoExtralight}{cgoLight}

  \begin{xtabular}{#2}
    #5
   \bottomrule
  \end{xtabular}
  \label{tabla:#4}
  \end{center}
}



\graphicspath{ {./Imagenes/} }

% Capítulos
\chapterstyle{bianchi}
\newcommand{\capitulo}[2]{
	\setcounter{chapter}{#1}
	\setcounter{section}{0}
	\setcounter{figure}{0}
	\setcounter{table}{0}
	\chapter*{\thechapter.\enskip #2}
	\addcontentsline{toc}{chapter}{\thechapter.\enskip #2}
	\markboth{#2}{#2}
}

% Apéndices
\renewcommand{\appendixname}{Apéndice}
\renewcommand*\cftappendixname{\appendixname}

\newcommand{\apendice}[1]{
	%\renewcommand{\thechapter}{A}
	\chapter{#1}
}

\renewcommand*\cftappendixname{\appendixname\ }

% Formato de portada
\makeatletter
\usepackage{xcolor}
\newcommand{\tutor}[1]{\def\@tutor{#1}}
\newcommand{\course}[1]{\def\@course{#1}}
\definecolor{cpardoBox}{HTML}{E6E6FF}
\def\maketitle{
  \null
  \thispagestyle{empty}
  % Cabecera ----------------
\noindent\includegraphics[width=\textwidth]{cabecera}\vspace{1cm}%
  \vfill
  % Título proyecto y escudo informática ----------------
  \colorbox{cpardoBox}{%
    \begin{minipage}{.8\textwidth}
      \vspace{.5cm}\Large
      \begin{center}
      \textbf{TFG del Grado en Ingeniería Informática}\vspace{.6cm}\\
      \textbf{\LARGE\@title{}}
      \end{center}
      \vspace{.2cm}
    \end{minipage}

  }%
  \hfill\begin{minipage}{.20\textwidth}
    \includegraphics[width=\textwidth]{escudoInfor}
  \end{minipage}
  \vfill
  % Datos de alumno, curso y tutores ------------------
  \begin{center}%
  {%
    \noindent\LARGE
    Presentado por \@author{}\\ 
    en Universidad de Burgos \\ a \@date{}\\
    Tutora: \@tutor{}\\
  }%
  \end{center}%
  \null
  \cleardoublepage
  }
\makeatother

\newcommand{\nombre}{Daniel Fernández Fernández} %%% cambio de comando

% Datos de portada
\title{TFG GII 23.23 Web de libros y sistema de clasificación}
\author{\nombre}
\tutor{Ana Serrano Mamolar}
\date{\today}

\begin{document}

\maketitle


\newpage\null\thispagestyle{empty}\newpage


%%%%%%%%%%%%%%%%%%%%%%%%%%%%%%%%%%%%%%%%%%%%%%%%%%%%%%%%%%%%%%%%%%%%%%%%%%%%%%%%%%%%%%%%
%\thispagestyle{empty}


%\noindent\includegraphics[width=\textwidth]{cabecera}\vspace{1cm}

%\noindent Dña. Ana Serrano Mamolar, profesora del departamento de ingeniería informática, área de Lenguajes y Sistemas Informáticos.

%\noindent Expone:

%\noindent Que el alumno D. \nombre, con DNI 71305558S, ha realizado el Trabajo final de Grado en Ingeniería Informática titulado título de TFG. 

%\noindent Y que dicho trabajo ha sido realizado por el alumno bajo la dirección del que suscribe, en virtud de lo cual se autoriza su presentación y defensa.

%\begin{center} %\large
%En Burgos, {\large \today}
%\end{center}

%\vfill\vfill\vfill

% Author and supervisor
%\begin{minipage}{0.45\textwidth}
%\begin{flushleft} %\large
%Vº. Bº. del Tutor:\\[2cm]
%Dña. Ana Serrano Mamolar,
%\end{flushleft}
%\end{minipage}
%\hfill
%\begin{minipage}{0.45\textwidth}
%\begin{flushleft} %\large
%Vº. Bº. del co-tutor:\\[2cm]
%D. Jesus Alberto San Martín Zapatero,
%\end{flushleft}
%\end{minipage}
%\hfill

%\vfill

% para casos con solo un tutor comentar lo anterior
% y descomentar lo siguiente
%Vº. Bº. del Tutor:\\[2cm]
%D. nombre tutor


%\newpage\null\thispagestyle{empty}\newpage




\frontmatter

% Abstract en castellano
\renewcommand*\abstractname{Resumen}
\begin{abstract}
La educación de los niños es fundamental para su correcto desarrollo y la posibilidad de un futuro brillante. 

En este aspecto, la literatura infantil constituye uno de los recursos más adecuados para la aproximación al conocimiento de realidades sociales y culturales. En el campo de la prehistoria es a menudo bastante común la reproducción de estereotipos de género que sin embargo la ciencia ya ha descartado definitivamente. Es muy importante por tanto para docentes y familias descubrir literatura infantil que no contribuyan a perpetuar estos estereotipos sino que muestren una realidad más alineada con los datos avalados por la ciencia. Con este enfoque hay diferentes investigaciones que abordan la evaluación de los libros de prehistoria y su adecuación en la etapa infantil.

En este trabajo, se propone una plataforma que sirva de catálogo disponible para cualquier docente o familia a la hora de escoger qué libros de prehistoria ofrecer a las niñas y niños así como ofrecer mecanismos de evaluación derivados de investigaciones para la auto-evaluación de nuevos títulos.

Adicionalmente, esta plataforma web incluye un apartado de administración que proporciona todas las herramientas necesarias para un desarrollo eficiente y colaboración en la integración de nuevos títulos. Entre las funcionalidades se encuentran la gestión de cuentas, permisos, copias de seguridad del catálogo y la búsqueda automática de libros en tres fuentes diferentes.

La web está disponible ininterrumpidamente en el siguiente enlace: \href{https://prehistoriaenigualdad.netlify.app/}{Prehistoria en igualdad}.
\end{abstract}

\renewcommand*\abstractname{Descriptores}
\begin{abstract}
Aplicación web, full-stack, web scraping, libros, flask, base de datos, angular, python
\end{abstract}

\clearpage

% Abstract en inglés
\renewcommand*\abstractname{Abstract}
\begin{abstract}
The education of children is essential for their correct development and the possibility of a bright future. 

In this aspect, children's literature constitutes one of the most appropriate resources for approaching knowledge of social and cultural realities. In the field of prehistory, the reproduction of gender stereotypes is often quite common, although science has already definitively discarded them. It is therefore very important for teachers and families to discover children's literature that does not contribute to perpetuating these stereotypes but rather shows a reality more aligned with data supported by science. With this approach, there are different investigations that address the evaluation of prehistory books and their suitability in the childhood stage.

In this work, a platform is proposed that serves as a catalog available to any teacher or family when choosing which prehistory books to offer to girls and boys, as well as offering evaluation mechanisms derived from research for the self-assessment of new titles. .

Additionally, this web platform includes an administration section that provides all the necessary tools for efficient development and collaboration in the integration of new titles. Features include account management, permissions, catalog backups, and automatic search for books in three different sources.

The website is available continuously at the following link: \href{https://prehistoriaenigualdad.netlify.app/}{Prehistoria en igualdad}. 

 
\end{abstract}

\renewcommand*\abstractname{Keywords}
\begin{abstract}
Web app, full-stack, web-scrapping, books, flask, database, angular, python 
\end{abstract}

\clearpage

% Indices
\tableofcontents

\clearpage

\listoffigures

\clearpage

\listoftables
\clearpage

\mainmatter
\capitulo{1}{Introducción}
«Durante más de siglo y medio, las interpretaciones que se han hecho de los restos arqueológicos han contribuido en gran medida a invisibilizar a las mujeres prehistóricas, sobre todo al reducir su importancia en la economía», según Marylène Patou-Mathis~\cite{CitaIntro}.

En el contexto actual, donde la digitalización de la información juega un papel crucial en numerosos sectores, este trabajo de fin de grado introduce una aplicación web innovadora enfocada en el ámbito específico de la literatura infantil prehistórica. Esta aplicación está diseñada para facilitar el almacenamiento, la gestión y la clasificación de libros infantiles que abordan la temática prehistórica, con un enfoque particular en la representación de los roles de género conforme a los descubrimientos científicos más recientes.

La literatura infantil prehistórica ofrece una ventana única al pasado, permitiendo a los jóvenes lectores explorar cómo vivían, interactuaban y se organizaban las sociedades antiguas. Sin embargo, es crucial que estas representaciones sean fieles a los avances científicos actuales en cuanto a los roles de género, evitando perpetuar estereotipos desfasados o inexactitudes históricas.

Nuestra aplicación aborda esta necesidad al proporcionar una plataforma donde los libros infantiles sobre la prehistoria pueden ser catalogados y mostrados según su precisión y representación de los roles de género. Esto incluye una evaluación de cómo cada libro representa las dinámicas de género en contextos prehistóricos, asegurando que reflejen los conocimientos y descubrimientos científicos más recientes.

Una característica distintiva de esta herramienta es su capacidad para destacar libros que promueven una comprensión informada y actualizada de los roles de género en la prehistoria. Esto es especialmente valioso para el personal de bibliotecas, docentes y familias que buscan ofrecer a los niños y niñas una perspectiva adecuada y didáctica, que fomente el pensamiento crítico y la comprensión de la diversidad y la igualdad de género desde una edad temprana.

Además, la aplicación no solo sirve como un repositorio de información, sino que también actúa como una guía para seleccionar material de lectura que esté alineado con los hallazgos científicos actuales, proporcionando así una base sólida para la educación y la sensibilización sobre los roles de género en diferentes épocas históricas, empezando por la prehistoria.

\subsection{Estructura de la memoria}
Este proyecto consta de dos documentos, una memoria y un documento de anexos, los cuales se complementan para documentar completamente este proyecto.
La estructura de la memoria es la siguiente:

\textbf{Introducción}:  Descripción inicial del problema y la solución creada. Estructura de la memoria y los materiales adjuntos.

\textbf{Objetivos del proyecto}: Listado de los objetivos a cumplir durante la realización del proyecto.

\textbf{Conceptos teóricos}: Explicación de los conceptos teóricos clave implementados en la solución propuesta.

\textbf{Técnicas y herramientas}: Técnicas y herramientas utilizadas o consideradas para el desarrollo del proyecto.

\textbf{Aspectos relevantes del desarrollo}: Consideración de los aspectos destacables que tuvieron lugar durante la realización del proyecto.

\textbf{Trabajos relacionados}: Proyectos que he utilizado para poder tomar referencias y desarrollar la aplicación.

\textbf{Conclusiones y líneas de trabajo futuras}: Conclusiones obtenidas tras la finalización del proyecto y opciones de desarrollo a futuro disponibles.

Además de la estructura de la memoria, procedo a comentar la estructura de los anexos:

\textbf{Plan del proyecto}: Planificación temporal del desarrollo del proyecto y estudio de viabilidad económico y legal del proyecto.

\textbf{Requisitos}: Se contempla el listado de requisitos funcionales así como el diagrama de casos de uso explicado.

\textbf{Diseño}: Se mencionan las diferentes partes del diseño de la aplicación, como el diseño de los datos y el diseño arquitectónico.

\textbf{Manual del programador}: Incluye los aspectos más importantes del código fuente y una guía detallada de los pasos a realizar en caso de querer continuar con el desarrollo.

\textbf{Manual de usuario}: Manual de usuario para el  manejo eficiente de la aplicación.


\subsection{Materiales Adjuntos}
A continuación se muestra un listado con los materiales adjuntos a este documento:
\begin{itemize}
    \item \textit{Frontend} desarrollado en Angular, utilizado para la interfaz de usuario y se encarga de realizar las llamadas necesarias a la API.
    \item API REST en Python que contiene el \textit{backend}. Contiene toda la lógica necesaria en base a los requisitos funcionales establecidos en los anexos.
    \item Aplicación web del proyecto Prehistoria en igualdad desplegada donde el equipo de administración ha comenzado con la carga de libros y datos reales.\footnote{Acceso a la \href{https://prehistoriaenigualdad.netlify.app/}{Web desplegada}}
    \item Repositorio del proyecto que contiene todo el desarrollo del proyecto.\footnote{Acceso al \href{https://github.com/DanielFernandezFdez/TFG\_GII\_23.23}{repositorio de GitHub}}
\end{itemize}
 
\capitulo{2}{Objetivos del proyecto}
En este apartado apartado se van a exponer los diferentes objetivos planteados inicialmente en el proyecto divididos en varios apartados.
\subsection{Objetivos generales}
\begin{itemize}
    \item Dar a conocer los resultados de investigaciones y metodologías desarrolladas por el Didáctica de la Historia y de las Ciencias Sociales de la Universidad de Burgos.
    \item Desarrollar una página web para la consulta de libros sobre la prehistoria aptos para todos los públicos en base a criterios científicamente fundamentados sobre roles de género.
    \item Proveer herramientas de gestión de la web para las personas que administren la página.
    \item Generar herramientas de seguridad para el control y limitación de las personas colaboradoras de la web.
    \item Almacenar todos los datos en una base de datos persistente y correctamente estructurada.
    \item Importar y exportar datos de la web con facilidad para asegurar su conservación, sincronización y uso de estudios externos.

\end{itemize}

\subsection{Objetivos técnicos}
\begin{itemize}
    \item Implementar \textit{web scraping} y consultas a APIs públicas, junto a una lógica para trabajar con los datos en el \textit{backend} de la web. Todo esto se utilizaría para la búsqueda en distintas fuentes bibliográficas.
    \item Crear una calculadora web que permita estimar en base a una metodología validada el porcentaje de realidad científica existente centrado en roles de género.
    \item Implementación de diferentes roles de uso de la aplicación para limitar el acceso, de manera dinámica, a ciertas acciones de administración de la página.
    \item Utilizar un sistema de control de versiones como GitHub.
    \item Utilizar metodologías ágiles como \textit{SCRUM} y herramientas de organización de trabajo como \textit{Kanban}.
    \item Implementar y validar el interfaz web utilizando el \textit{framework} de Python Flask.
    \item Publicación del proyecto en un servicio como Netlify o Render para su acceso público.

\end{itemize}

\subsection{Objetivos personales}
\begin{itemize}
    \item Formarme en el desarrollo web.
    \item Formarme en el desarrollo de una aplicación full-stack.
    \item Brindar herramientas que ayuden a mejorar la educación.
    \item Aplicar los conocimientos adquiridos durante la carrera y las prácticas curriculares.
\end{itemize}


\capitulo{3}{Conceptos teóricos}

Dentro de este proyecto existen elementos o implementaciones que pueden llegar a tener una mayor complejidad teórica. Por ese mismo motivo, en este apartado se van a detallar en diferentes secciones los aspectos más complejos que han surgido durante el desarrollo.

\section{Web scraping}
El \textit{web scraping}~\cite{Webscraping} es una técnica usualmente utilizada para poder obtener de forma automática y estructurada los contenidos de las páginas web para su utilización en diferentes fines. 
En el caso de este proyecto, para obtener los datos detallados de los libros deseados.

Esta técnica puede resultar muy beneficiosa debido a que se pueden obtener grandes volúmenes de datos minimizando las posibilidades de error frente a una búsqueda manual, la cuál es más lenta y costosa.

Aunque este sistema de obtención de datos sea muy útil y eficiente, antes de aplicarlo a una página web para la obtención de los datos, es imprescindible comprobar que esta página nos da la autorización para realizarlo.

Para realizar esta comprobación de manera legal y sencilla, y así evitar posibles problemas legales, es fundamental consultar el archivo \textit{robots.txt}~\cite{Robots.txt} de la página web. Este archivo es un protocolo de exclusión de robots que define a qué URLs de una página web pueden acceder los rastreadores o bots. Al acceder a la URL de la página web \textit{/robots.txt}, se puede verificar si la web permite o restringe el acceso a ciertos servicios y redireccionamientos para realizar \textit{web scraping}.

El archivo \textit{robots.txt} proporciona una lista de reglas que indican a los rastreadores web qué partes del sitio deben evitar, ayudando así a prevenir accesos no autorizados y proteger la integridad del sitio web. Este protocolo especifica, mediante directivas, qué áreas de la página web no deben ser rastreadas y puede incluir secciones como \textit{'Disallow'} para bloquear acceso a determinadas rutas o archivos y \textit{'Allow'} para permitir acceso a otras partes.

\begin{figure}[htbp]
    \centering
    \includegraphics[width=0.5\linewidth]{Imagenes/RobotsAgapea.png}
    \caption{Archivo robots de Agapea}
    \label{Archivo robots de Agapea}
\end{figure}
\FloatBarrier
En la Figura \ref{Archivo robots de Agapea} podemos observar el archivo \textit{robots.txt} de la web de la librería Agapea \footnote{\href{https://www.agapea.com/robots.txt}{Enlace a robots.txt}}, la cual hemos utilizado en el desarrollo de este proyecto. Tal como se muestra, las primeras líneas hacen referencia a distintos rastreadores y servicios específicos los cuales no pueden acceder a las URL marcadas en el apartado de "disallow" (En este caso, los servicios que aparecen no tienen permisos para acceder a ninguna URL perteneciente a esta web).

Debajo de estas primeras líneas observamos otro grupo de líneas que son las que indican cuáles son las URL a las que no está permitida la entrada o utilización de ningún tipo de servicio o rastreador, indistintamente del tipo que sea.

Otras formas de bloqueo existentes que podrían afectar a la utilización de esta herramienta son las siguientes:
\begin{itemize}
    \item \textit{CAPTCHAs}:  Bloqueos simples para poder distinguir entre un robot y un humano.
    \item Limite de peticiones: Estableciendo un límite no se podrían obtener los datos de una forma tan rápida, ya que la propia web lo bloquea al superar el límite de peticiones.
    \item Bloqueo por IP: Las páginas web pueden contener una lista negra de usuarios que no puedan acceder a la web, esto se hace a través de listas de IPs.
\end{itemize}

Una vez se establecen las limitaciones y se tienen en cuenta en la aplicación del web scraping, se pueden utilizar diferentes herramientas para implementarlo, como bibliotecas o frameworks, que permiten un desarrollo correcto de análisis de los datos de las páginas web deseadas. En secciones posteriores se describen las utilizadas para este proyecto.


\section{Json Web Token (JWT)}
La tecnología JWT es una tecnología incluida en el estandar RFC 7519~\cite{JWT_Estandar} que permite enviar y recibir información de forma compacta y rápida dentro de un elemento JSON. Este elemento es muy seguro debido a que se encuentra firmado digitalmente.

\subsection{Estructura de un token JWT}
Para poder realizar las firmas y generar las claves, se suele utilizar una clave privada y única a partir de la cual se generan el resto, comúnmente utilizando el algoritmo SHA256.

En el caso de este proyecto, esta tecnología se ha utilizado para poder limitar el acceso a personas no autorizadas a los recursos de la API que han de ser privados y solo gestionados por aquellas personas que tengan la autorización pertinente.

Un token de acceso JWT tiene un patrón de construcción, ya que se divide en 3 partes separados por "." , los cuales se describirán a continuación~\cite{JWT_Info}.
\begin{itemize}
    \item Header
    \item Payload
    \item Signature
\end{itemize}

A continuación, se va a ir analizando un ejemplo de JWT, para ello utilizaremos un debugger de JWT~\cite{JWT_Debugger}:

Header:

\texttt{eyJhbGciOiJIUzI1NiIsInR5cCI6IkpXVCJ9}

En el header indica principalmente dos elementos, el tipo de algoritmo que se ha usado para generarlo, y que tipo de token es. En este caso se ha utilizado HS256.

\begin{figure}[htbp]
    \centering
    \includegraphics[width=0.5\linewidth]{Imagenes/HeaderJWT.png}
    \caption{Header}
    \label{Header}
\end{figure}
\FloatBarrier

Payload:

\texttt{eyJzdWIiOiIxMjM0NTY3ODkwIiwibmFtZSI6IkRhbmllbCBGZXJuw6FuZGV6IiwiaWF}

\texttt{0IjoxNTE2MjM5MDIyfQ}

El apartado del payload contiene informaciones como el nombre de usuario, la fecha de la creación del token, y a quiín se refiere el token.

\begin{figure}[htbp]
    \centering
    \includegraphics[width=0.5\linewidth]{Imagenes/JWT Payload.png}
    \caption{Payload}
    \label{Payload}
\end{figure}
\FloatBarrier

Signature:

\texttt{RrvHMGQTYf15y5\_WgdoH1HULZDRYukEmb5iMgAIwW0k}

Este último apartado es la firma del token, este apartado se forma con el algoritmo especificado en el apartado de \textit{header} junto con una clave secreta que se defina por el administrador y los apartados anteriores codificados.

\begin{figure}[htbp]
    \centering
    \includegraphics[width=0.5\linewidth]{Imagenes/JWTFirma.png}
    \caption{Signature}
    \label{Signature}
\end{figure}
\FloatBarrier

\subsection{Funcionamiento de un JWT }
Cada vez que una persona usuaria desea acceder a un recurso protegido con JWT, tiene que realizar una petición al \textit{backend} para que le retorne una clave nueva, si tenía una existente, no es necesario hacer la petición siempre y cuando ese token no expire.
En el caso de este proyecto, el token está configurado para proteger información de los usuarios registrados, además de no permitir la modificación ni obtención del catálogo. El token para poder realizar estas acciones se obtiene de manera invisible al usuario cuando realiza la operación de inicio de sesión.

Al obtener este token, es almacenado en el local storage del navegador, por lo que el usuario no necesita en ningún momento manipularlo ni editarlo.

Una vez obtenido, al realizar una petición al \textit{backend} será necesario incluir en la cabecera de la llamada el token generado. Esto se realiza internamente desde el \textit{frontend}, obteniendo el token de la zona de almacenamiento e incluyéndolo. Es importante mencionar que no siempre es necesario mandar al \textit{backend} el token, ya que en este proyecto no todas las llamadas lo requieren.

Tras este envío al servidor, el \textit{backend} realiza dos comprobaciones:
\begin{itemize}
    \item Fecha de expiración
    \item Validez del token
\end{itemize}
Si ambas comprobaciones resultan satisfactorias, el servidor responde a la petición correctamente.

\section{Estimación de la adecuación didáctica de los libros}
Este proyecto contiene un apartado que genera una estimación en un rango de 0 a 100 indicando al usuario la calidad de un libro introducido en términos de realidad científica en perspectiva de género. Este proceso de estimación se ha basado en la metodología publicada por Jesús Alberto San Martín Zapatero y Delfín Ortega Sánchez~\cite{san2022atribuciones}. Para realizar estos cálculos, el \textit{frontend} muestra al usuario un formulario donde tiene que rellenar los siguientes campos:
\begin{itemize}
    \item Campo de selección si existe masculino genérico.
    \item Cuatro campos donde se deben de introducir el número de personas adultas y menores existentes en el libro.
    \item Por cada uno de los géneros, se seleccionan las actividades que realizan de entre las opciones de una lista.
    \item La ubicación en la que se ambienta el libro.

\end{itemize}
Dentro de este formulario todos los campos son obligatorios exceptuando el de los menores, ya que es posible que no aparezcan, en cuyo caso la estimación se reajusta para adaptarse. Más detalles de la metodología de evaluación pueden encontrarse en ~\cite{san2022atribuciones}.

Una vez obtenidos estos datos se envían al \textit{backend} que va realizando las siguientes comprobaciones para ir ajustando la estimación.
\begin{enumerate}
    \item Obtiene el campo del masculino genérico y si la respuesta es negativa, se añaden 20 puntos a la nota final.
    \item En el caso de los contadores de personas, se realiza una media de equilibrio entre géneros tanto para personas adultas como para menores de forma independiente para obtener cómo de igualados están los dos géneros.  Idealmente tendrían que aparecer en la misma proporción o muy similar, lo que daría un resultado de 15 puntos para los personas adultas y 15 puntos para los menores. Si la proporción se encontrase desbalanceada, la nota sería menor ya sea en personas adultas o en menores.

En el caso de que el número total de menores existentes en el formulario sea el valor 0, la media se ajusta para no tener en cuenta este grupo y valorar sobre 30 puntos a los personas adultas.
\item Las actividades candidatas se encuentran contenidas en una tabla de la base de datos a las cuales los responsables de la administración pueden entrar y variar las actividades existentes y realizar una gestión completa. Por lo que, al llegar al \textit{backend}, los elementos seleccionados se cruzan con todos las actividades existentes y se obtienen de las comunes si le corresponde dar una puntuación extra o no por realizar esa acción (Una acción poco común para ese género, por lo tanto se premia). En base a estas comprobaciones se asignan hasta 20 puntos de la nota final de estimación.
\item Finalmente, el apartado de ubicaciones es un selector con unos valores fijos que se asignan y se envían en base a la elección del usuario. Cuanta más diversidad exista en la ubicación del libro, más cerca estará la puntuación de los 30 puntos de este apartado.

\end{enumerate}

Una vez terminado el análisis completo de los datos introducidos, se realiza una suma de todas las notas de cada apartado y se envía en escala 0-100, donde 0 es un libro muy poco adecuado y 100 es un libro que se corresponde a la perfección con la evidencia científica. Tras esto, el \textit{frontend} muestra al usuario un botón donde puede registrar su respuestas para que posteriormente los personas colaboradoras y responsables de la administración de la web puedan analizarlo y considerar agregar ese libro de cuya estimación ha sido realizada.





\capitulo{4}{Técnicas y herramientas}


\section{Framework utilizado para el desarrollo del proyecto}

\subsubsection{Herramienta Elegida: Flask}

Flask es un microframework para Python que ha sido diseñado para facilitar el desarrollo de aplicaciones web. A diferencia de otros frameworks más complejos y rígidos, Flask proporciona la flexibilidad necesaria para adaptar la estructura del proyecto a las necesidades específicas de la web de libros de la prehistoria. Ofrece un conjunto de herramientas esenciales que simplifican procesos como el enrutamiento, la gestión de sesiones y la integración con bases de datos.

\subsubsection{Funcionalidades y Ventajas de Flask}

La elección de Flask se debe a su capacidad para manejar tanto aspectos básicos como avanzados del desarrollo web:

\begin{itemize}
    \item Desarrollo Ágil: Flask permite un rápido desarrollo y prototipado, lo que es ideal para los ciclos de iteración del TFG.
    \item Simplicidad y Flexibilidad: Su simplicidad  facilita la comprensión del código, lo que es fundamental para un TFG, donde la claridad y la documentación son clave.
    \item Ecosistema Completo: Existe una amplia variedad de extensiones disponibles que permiten añadir funcionalidades adicionales según sea necesario, sin sobrecargar el núcleo de la aplicación.
    \item Licencia: Flask está disponible bajo la licencia BSD, una licencia de software libre que permite la reutilización y distribución del código con pocas restricciones.
\end{itemize}
 


\section{Desarrolllo de prototipo Web}

\paragraph{Herramienta Elegida: Justinmind}

Justinmind es una herramienta de prototipado interactiva utilizada en este proyecto para diseñar la interfaz de la web de libros de la prehistoria. Fue elegida por su facilidad de uso, y amplia biblioteca de widgets.

\paragraph{Ventajas de Justinmind}

\begin{itemize}
    \item Interactividad: Simulación realista de la interfaz final de usuario
    \item Versatilidad: Adecuada para prototipos de baja y alta fidelidad.
\end{itemize}

\paragraph{Uso en el Proyecto}

Justinmind ha sido fundamental para definir y validar la experiencia del usuario, realizar pruebas de usabilidad y facilitar la comunicación del diseño entre los desarrolladores y los stakeholders.

 



\section{Análisis Comparativo entre Google Books API y Amazon Books API}

\subsection{Accesibilidad y Documentación}
\begin{itemize}
    \item \textbf{Google Books API:} Ofrece accesibilidad superior y documentación detallada. Proporciona una clave de API gratuita con un límite de 1,000 solicitudes diarias.
    \item \textbf{Amazon Books API:} Requiere afiliación a Amazon Advertising API y está orientada hacia usuarios con propósitos comerciales. La documentación es robusta pero menos intuitiva.
\end{itemize}

\subsection{Amplitud de Datos Disponibles}
\begin{itemize}
    \item \textbf{Google Books API:} Acceso a más de 25 millones de libros con información extensa, ideal para proyectos educativos o bibliotecarios.
    \item \textbf{Amazon Books API:} Proporciona datos orientados a ventas y reseñas, incluyendo rankings y precios, útil para análisis de mercado.
\end{itemize}

\subsection{Facilidad de Integración y Uso}
\begin{itemize}
    \item \textbf{Google Books API:} Fácil integración gracias a su estructura basada en REST y compatibilidad con múltiples lenguajes de programación.
    \item \textbf{Amazon Books API:} Requiere comprensión avanzada de las API de Amazon y sus requisitos de autenticación, representando una curva de aprendizaje más pronunciada.
\end{itemize}

\subsection{Restricciones de Uso y Limitaciones}
\begin{itemize}
    \item \textbf{Google Books API:} Tiene limitaciones en el número de solicitudes diarias, pero generalmente suficientes para muchos proyectos.
    \item \textbf{Amazon Books API:} Limitaciones más estrictas en cuanto a la frecuencia de las solicitudes y acceso a ciertos datos.
\end{itemize}

\section{Justificación para la Elección de Google Books API}
La elección de la API de Google Books se justifica por su accesibilidad, amplia gama de datos bibliográficos, facilidad de integración, y cuota generosa de solicitudes gratuitas. Esto la hace ideal para un entorno académico o de investigación, donde los recursos pueden ser limitados y la facilidad de uso es crucial.










\capitulo{5}{Aspectos relevantes del desarrollo del proyecto}

\section{Metodologías Aplicadas}

Desde el inicio del proyecto se tenía muy claro que se iba a proceder de la manera más ordenada y profesional posible. Para poder realizar exitosamente este proceso recurrimos diferentes herramientas que nos permitiesen realizar un correcto desarrollo de este proyecto.
Fundamentalmente nos centramos en aplicar metodologías Ágiles.  Este concepto se podría definir como el conjunto de reglas y técnicas aplicadas a ciclos de trabajo de duración reducida. Esto permite tener una mayor flexibilidad durante el proyecto y una correcta entrega continua que nos permite la máxima colaboración con el cliente, en este caso el profesor Jesús Alberto San Martín Zapatero.
Debido a que la metodología ágil engloba a diferentes tipos de las mismas, a continuación se mencionan las usadas.
\begin{itemize}
    \item Kanban:
\end{itemize}
 La metodología Kanban consiste en poder informarte del estado de las tareas del proyecto de una forma visual y rápida, por lo que, con una rápida visualización, podemos saber el estado de cada una de las tareas existentes en el tablero.
 \begin{itemize}
     \item Scrum
 \end{itemize}
Esta metodología principalmente se posiciona en realizar ciclos de trabajos (Sprints) con una duración fija, en la que al finalizar se realiza una entrega del proyecto. 
En este caso se han realizado Sprints de 2 semanas con una reunión al finalizar el Sprint para poder debatir acerca de la entrega realizada y realizar las propuestas de trabajo para el siguiente Sprint.
Una vez generadas esas propuestas se transformaban en tareas que se incluían en el tablero Kanban para realizar un seguimiento a tiempo real del progreso del Sprint.
\begin{itemize}
    \item Lean
\end{itemize}
Esta metodología es posible que no sea tan frecuente como las dos anteriores, pero se fundamenta en la mejora continua y eliminar todos los lastres de tiempo posible en el proyecto. Esto permitía tener más tiempo para poder mejorar la calidad lo máximo posible al disponer de más tiempo para centrarnos en los detalles


\section{Inicio del proyecto}
Como primeros pasos para la integración del proyecto, se realiza un estudio para considerar el lenguaje que más se pudiera adaptar como se ha mencionado anteriormente como el Framework que nos pudiera permitir un desarrollo exitoso desde un inicio. De manera adicional, se determina como idea inicial obtener los datos a través de la realización de peticiones a una API de la que se obtuviese la información de los libros deseados.

Tras tomar esas decisiones iniciales, comienzo a investigar la estructura del Framework y los lenguajes HTML y CSS, ya que mi experiencia anterior con estas herramientas era muy limitada.

Una vez realizadas las investigaciones pertinentes, se genera un primer prototipo de página web con la estructura inicial y los elementos fundamentales. 
Antes de comenzar a trabajar con la API de Google, se crea una web básica con un sistema de gestión de libros utilizando una base de datos que permite realizar las siguientes operaciones de manera manual:

\begin{itemize}
    \item Barra de navegación que permite cambiar de página.
    \item Sistema de Login para gestionar la aparición de botones ocultos.
    \item Consultar todos los libros existentes en la base datos.
    \item Realizar búsquedas de los libros existentes en la base de datos en base al título o al ISBN.
    \item Agregar un libro a la base de datos.
    \item Editar un libro de la base de datos.
    \item Eliminar un libro de la base de datos.

\end{itemize}

El sistema de Login se estableció para no permitir que ningún usuario que no tuviese credenciales pudiera realizar modificaciones de la base de datos. En el caso de que ejecutaran la URL exacta de la función, la protección de seguridad les llevaría a la pantalla de Inicio de Sesión.

\begin{figure}[h]
    \centering
    \includegraphics[width=0.9\textwidth]{Imagenes/Inicio_Sesion.png}
    \caption{Inicio de sesión}
    \label{fig:Inicio de sesión}
\end{figure}


\section{Desarrollo del proyecto}

Una vez presentado y debatido este primer prototipo de página web en el Sprint, se decide continuar de esta base agregando más elementos para añadirle funcionalidades. De manera adicional, se comienza a realizar un prototipo web para ir documentando el esquema de secciones existente en este proyecto. Tal como se comenta en la sección de Técnicas y Herramientas,  el prototipo web se realiza con el programa Justinmind.

Además, se agrega otra funcionalidad considerada como esencial, una página que al seleccionar el libro te muestre toda la información disponible en la base de datos para todo usuario interesado en obtener datos.
Para ello, en cada libro mostrado en el catálogo se pone a disposición un botón que te redirige a esta página.

Durante esa reunión se propone la idea de incluir un apartado especifico para los administradores que permite importar y exportar en un archivo CSV todos los datos de cada libro contenidos en la base de datos. 
Esta idea se concibe principalmente por dos grandes motivos
\begin{itemize}
    \item Tener la oportunidad de volcar datos más rápido desde el archivo CSV y posteriormente cargar el contenido del fichero en la base de datos.
    \item Poder tener un duplicado de la base de datos por si el contenido de la misma quedase inaccesible o se realizara un cambio indeseado, lo cuál se solucionaría restableciendo el catálogo con el archivo de seguridad descargado.

\end{itemize}
Aunque esta idea inicialmente funcionara. Al realizar pruebas se detectó un error durante la importación.
El error consistía en que un administrador pudiera colocar el signo "," para separar dos elementos como por ejemplo dos ISBN o que se encontrase en el contenido de la descripción del libro, lo cual generaba errores a la hora de transformar los campos a columnas al tener ese mismo signo como delimitador. Por ese motivo se propuso la idea de cambiar el delimitador por defecto a el signo ";".

Tras realizar estas correcciones de errores iniciamos el proceso de obtener la información  de los libros de forma automática Aunque de manera inicial se valoró la opción de utilizar únicamente la API de Google Books,  se consideró interesante para  dar más fuentes de información y aumentar la dificultad técnica del proyecto la utilización de Web scraping. 
El  web scraping realizado se aplicó sobre 2 páginas web las cuales tienen un catálogo de libros muy amplio. Estas webs son \href{https://www.amazon.es/}{Amazon} y \href{https://www.agapea.com/}{Agapea}. Estos tres proveedores nos permiten que en base a un ISBN o un Título, internamente se lancen unos procesos que busquen en los proveedores con las diferentes técnicas mostradas para que se pueda seleccionar el proveedor que nos de la información más precisa. 
En el caso de que se quisiera agregar un libro automáticamente pero se desee personalizar y usar una mezcla de los 3 proveedores, se dispone de un sistema de desplegables que nos permite realizar esta acción.

Durante el desarrollo en este espacio de tiempo, se considera que sería necesario implementar un sistema que nos permitiese proteger el catálogo de una importación que pueda encontrarse corrupta o con datos erróneos.
Por lo que antes de realizar la importación, aparece una ventana modal que recuerda al usuario que debería de exportar el catálogo actual por seguridad.
\capitulo{6}{Trabajos relacionados}

Durante la realización de este proyecto, se han tomado referencias de otros trabajos para poder tomar ideas y moldearlas de manera que se genere un producto completo y de calidad. A continuación, se detallan los trabajos y programas de los que se han tomado referencias o que se encuentran relacionados en este ámbito.

\section{Atribuciones de género y construcción de identidades en la literatura infantil sobre prehistoria }

El estudio de Alberto San Martín Zapatero y Delfín Ortega-Sánchez~\cite{san2022atribuciones}, titulado 'Atribuciones de género y construcción de identidades en la literatura infantil sobre prehistoria', examina cómo los libros infantiles dedicados a la prehistoria presentan estereotipos de género. Publicado en el Bellaterra Journal of Teaching \& Learning Language \& Literature, este trabajo utiliza técnicas de análisis de contenido para evaluar textos e ilustraciones, revelando que a menudo se distorsionan los roles de género y se alejan de los últimos avances científicos.

Este proyecto se inspira en este estudio para desarrollar su metodología de análisis. La herramienta del estimador del proyecto evalúa de manera objetiva la presencia de sesgos de género en los libros infantiles, adaptando y ampliando las categorías de análisis propuestas en el estudio de San Martín Zapatero y Ortega-Sánchez.

Ambos trabajos comparten el objetivo de alcanzar un discurso inclusivo en la literatura infantil sobre prehistoria y basado en evidencias científicas, proporcionando recursos educativos que reflejen una visión más justa de la prehistoria. Esta colaboración y adaptación de metodologías refuerza la importancia de revisar y actualizar los materiales educativos para eliminar los estereotipos de género.

\section{Discord}

Discord\footnote{Web de \href{https://discord.com/}{ Discord}} es una red social de comunicación que sido una gran referencia para la parte lógica, ya que se ha tomado como referencia para la lógica de permisos del personal de administración. Esta parte ha sido desarrollada de manera similar, donde en un grupo el administrador puede crear los roles que se necesiten y aplicar unos permisos personalizados para los usuarios con esos roles.

\begin{figure}[h]
\centering
\includegraphics[width=0.6\linewidth]{Imagenes/Discord.png}
\caption{Permisos Discord}
\label{Permisos Discord}
\end{figure}
\FloatBarrier
\capitulo{7}{Conclusiones y Líneas de trabajo futuras}

En este apartado se van a tratar las conclusiones obtenidas de este trabajo junto a posibles líneas de trabajo futuras por donde se puede continuar el proyecto.

\section{Conclusiones}
A continuación se detallan las conclusiones de este proyecto:
\begin{itemize}
    \item El objetivo inicial se ha cumplido de manera satisfactoria e incluso pudiendo completar la aplicación con elementos adicionales. Al tener este proyecto finalmente desplegado, tanto docentes como familias tienen la oportunidad de obtener información que beneficie a la educación y poder participar descubriendo cómo de adecuado según los estudios científicos es un libro en lo relativo a sesgos de género.
    \item Las tecnologías propuestas inicialmente no han sido las que se han utilizado en todos los casos. Durante la realización del proyecto se decidió realizar un cambio de rumbo y pasar el \textit{frontend} de Flask a Angular, permitiendo así desarrollar el proyecto en un entorno con más herramientas y posibilidades de escalar en el futuro.
    \item Gracias a la parte de investigación y necesidad de enfrentarme a diferentes retos completamente desconocidos, se ha aprendido a realizar investigaciones exhaustivas de una herramienta y comenzar a tener un criterio en relación a estudios de viabilidad.
    \item Ha sido complejo intentar realizar estimaciones acertadas, ya que han existido ocasiones en los que la tecnología ha sido completamente desconocida y no se podían realizar estimaciones fiables. Para estos casos, la organización en \textit{sprints} ha sido muy beneficiosa para adaptar la carga de trabajo a las circunstancias.
    \item Se ha podido obtener mucho conocimiento acerca del trato con el cliente, ya que al tener reuniones y definir las especificaciones que debía de tener el producto, se generaban dos visiones totalmente distintas, las cuales había que unificar para encontrar un resultado acorde a la petición del cliente.
\end{itemize}

\section{Líneas de trabajo futuras}
A continuación se muestran posibles líneas de trabajo por las que se puede desarrollar este proyecto:
\begin{itemize}
    \item Una línea de trabajo interesante sería el desarrollo de filtros para ordenar los libros y guardar la configuración para guardar esa decisión.
    \item Otra opción que podría ser interesante implementar, sería mejorar el \textit{web scraping}  mostrando un catálogo completo de opciones por fuente detectando sólo los elementos que sean libros.
    \item En un futuro se podría perfeccionar la API en relación a los permisos para poder permitir a terceros utilizar esa API para realizar sus propias integraciones.
\end{itemize}


\bibliographystyle{plain}
\bibliography{bibliografia}

\end{document}