\apendice{Anexo de sostenibilización curricular}

\section{Introducción}
Este proyecto se alinea significativamente con varios Objetivos de Desarrollo Sostenible (ODS)~\cite{ODS} definidos por Naciones Unidas en 2015. Se trata de unos objetivos globales para erradicar la pobreza, proteger el planeta y asegurar la prosperidad para todos como parte de una nueva agenda de desarrollo cuyas metas pretenden alcanzarse en 2030. A continuación, se muestran las formas en que este proyecto contribuye a los ODS seleccionados.

\section{Objetivo número 4: Educación de calidad}
La literatura infantil puede mediar en la educación para la igualdad de género.  Este trabajo pretende facilitar la evaluación de literatura mediante el análisis de las atribuciones de género y los mecanismos de construcción identitaria en libros informativos y de ficción sobre prehistoria. Mediante la identificación de tareas desempeñadas por hombres y mujeres en el espacio social, la presencia o ausencia de rasgos conductuales referidos a la sumisión o a la dominación, conocer la frecuencia y grado de visibilidad o invisibilidad de hombres y mujeres en cada uno de los textos e imágenes, y el uso gramatical del masculino genérico para referirse a todas las
personas que conforman un colectivo humano. 

El proyecto ofrece un catálogo de libros infantiles, los cuales han sido evaluados en términos anteriormente comentados, lo que permite al personal docente y familias seleccionar libros que cumplan con un rigor científico acerca de los roles de género históricos o que al menos no perpetúa estereotipos de género que los estudios científicos ya han descartado desde hace tiempo. Al identificar y promover libros que representen de manera justa y precisa los roles de género, se fomenta una educación que refuerza los valores de igualdad y diversidad desde una edad temprana.

\subsection{Impacto en la práctica}
\begin{itemize}
    \item Formación docente: Proporciona al personal de educación una herramienta para evaluar y escoger libros para mejorar la enseñanza utilizando bases científicas. Esto facilita la tarea al personal docente al seleccionar materiales que no perpetúan estereotipos de género
    \item Participación familiar: Se involucra a las familias en la elección de libros y su proceso de selección en términos de calidad.
    \item Actualización continua: La herramienta permite actualizaciones periódicas basadas en nuevos estudios y hallazgos, asegurando que los recursos educativos se mantengan relevantes y científicamente precisos.
\end{itemize}

\section{Objetivo número 5: Igualdad de Género}
La aplicación web creada se centra en detectar y evaluar los estereotipos de género en los libros infantiles de la prehistoria. Al realizar esta acción, contribuye a la promoción de la igualdad de género desde una edad temprana, educando a niños y niñas con ejemplos equitativos y sin sesgos.

\subsection{Impacto en la práctica}
\begin{itemize}
    \item Eliminación de estereotipos: Al proporcionar análisis de los libros, esta herramienta contribuye a eliminar los estereotipos de género en la educación.
    \item Empoderamiento de niñas y mujeres: Este proyecto promueve una representación real y justa de la mujeres en la época de la prehistoria, lo que puede ayudar a inspirar a las niñas a interesarse por campos tradicionalmente ocupados por los hombres.
    \item Aumento de la sensibilización: Fomenta la concienciación sobre la importancia de la igualdad de género utilizando materiales educativos.
    \item Desarrollo de materiales educativos: Fomenta la creación de nuevos materiales educativos que sean inclusivos y equitativos, ayudando a reducir las brechas de género en la educación.
\end{itemize}


\section{Objetivo número 13: Acción por el clima}
Indirectamente, este proyecto contribuye a la acción por el clima al mejorar la eficiencia de la selección de los mejores materiales educativos. Al ayudar a identificar libros de calidad, tanto docentes como familias pueden escoger los mejores libros, evitando así comprar libros sin sesgos de género. Esto puede reducir el consumo de recursos y la huella de carbono asociada a su producción y distribución.

\subsection{Impacto en la práctica}
\begin{itemize}
    \item Consumo responsable: Indirectamente se fomenta un consumo más responsable de recursos educativos, seleccionando en un menor número de intentos la elección del mejor libro para la educación.
    \item Reducción de desperdicios: Al elegir los libros correctos desde el principio, se evita el desperdicio de recursos naturales y energía utilizados en la producción de libros que no cumplen con los estándares educativos.
\end{itemize}

\section{Conclusión}
Tras todo lo mostrado anteriormente, el proyecto desarrollado no solo cumple con su objetivo principal de ayudar a mejorar la enseñanza de la prehistoria libre de perpetuar roles de género inadecuados, si no que también contribuye a 3 de los Objetivos de Desarrollo Sostenible. A través de su implementación cuidadosa, la aplicación web demuestra cómo los objetivos globales pueden ser avanzados mediante pequeñas iniciativas educativas.