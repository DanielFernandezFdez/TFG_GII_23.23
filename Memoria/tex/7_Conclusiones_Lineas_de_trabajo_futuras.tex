\capitulo{7}{Conclusiones y Líneas de trabajo futuras}

En este apartado se van a tratar las conclusiones obtenidas de este trabajo junto a posibles líneas de trabajo futuras por donde se puede continuar el proyecto.

\section{Conclusiones}
A continuación se detallan las conclusiones de este proyecto:
\begin{itemize}
    \item El objetivo inicial se ha cumplido de manera satisfactoria e incluso pudiendo completar la aplicación con elementos adicionales. Al tener este proyecto finalmente desplegado, tanto docentes como familias tienen la oportunidad de obtener información que beneficie a la educación y poder participar descubriendo cómo de adecuado según los estudios científicos es un libro en lo relativo a sesgos de género.
    \item Las tecnologías propuestas inicialmente no han sido las que se han utilizado en todos los casos. Durante la realización del proyecto se decidió realizar un cambio de rumbo y pasar el \textit{frontend} de Flask a Angular, permitiendo así desarrollar el proyecto en un entorno con más herramientas y posibilidades de escalar en el futuro.
    \item Gracias a la parte de investigación y necesidad de enfrentarme a diferentes retos completamente desconocidos, se ha aprendido a realizar investigaciones exhaustivas de una herramienta y comenzar a tener un criterio en relación a estudios de viabilidad.
    \item Ha sido complejo intentar realizar estimaciones acertadas, ya que han existido ocasiones en los que la tecnología ha sido completamente desconocida y no se podían realizar estimaciones fiables. Para estos casos, la organización en \textit{sprints} ha sido muy beneficiosa para adaptar la carga de trabajo a las circunstancias.
    \item Se ha podido obtener mucho conocimiento acerca del trato con el cliente, ya que al tener reuniones y definir las especificaciones que debía de tener el producto, se generaban dos visiones totalmente distintas, las cuales había que unificar para encontrar un resultado acorde a la petición del cliente.
\end{itemize}

\section{Líneas de trabajo futuras}
A continuación se muestran posibles líneas de trabajo por las que se puede desarrollar este proyecto:
\begin{itemize}
    \item Una línea de trabajo interesante sería el desarrollo de filtros para ordenar los libros y guardar la configuración para guardar esa decisión.
    \item Otra opción que podría ser interesante implementar, sería mejorar el \textit{web scraping}  mostrando un catálogo completo de opciones por fuente detectando sólo los elementos que sean libros.
    \item En un futuro se podría perfeccionar la API en relación a los permisos para poder permitir a terceros utilizar esa API para realizar sus propias integraciones.
\end{itemize}