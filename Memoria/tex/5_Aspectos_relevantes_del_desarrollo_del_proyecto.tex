\capitulo{5}{Aspectos relevantes del desarrollo del proyecto}

\section{Metodologías Aplicadas}

Desde el inicio del proyecto se tenía muy claro que se iba a proceder de la manera más ordenada y profesional posible. Para poder realizar exitosamente este proceso recurrimos diferentes herramientas que nos permitiesen realizar un correcto desarrollo de este proyecto.
Fundamentalmente nos centramos en aplicar metodologías Ágiles.  Este concepto se podría definir como el conjunto de reglas y técnicas aplicadas a ciclos de trabajo de duración reducida. Esto permite tener una mayor flexibilidad durante el proyecto y una correcta entrega continua que nos permite la máxima colaboración con el cliente, en este caso el profesor Jesús Alberto San Martín Zapatero.
Debido a que la metodología ágil engloba a diferentes tipos de las mismas, a continuación se mencionan las usadas.
\begin{itemize}
    \item Kanban:
\end{itemize}
 La metodología Kanban consiste en poder informarte del estado de las tareas del proyecto de una forma visual y rápida, por lo que, con una rápida visualización, podemos saber el estado de cada una de las tareas existentes en el tablero.
 \begin{itemize}
     \item Scrum
 \end{itemize}
Esta metodología principalmente se posiciona en realizar ciclos de trabajos (Sprints) con una duración fija, en la que al finalizar se realiza una entrega del proyecto. 
En este caso se han realizado Sprints de 2 semanas con una reunión al finalizar el Sprint para poder debatir acerca de la entrega realizada y realizar las propuestas de trabajo para el siguiente Sprint.
Una vez generadas esas propuestas se transformaban en tareas que se incluían en el tablero Kanban para realizar un seguimiento a tiempo real del progreso del Sprint.
\begin{itemize}
    \item Lean
\end{itemize}
Esta metodología es posible que no sea tan frecuente como las dos anteriores, pero se fundamenta en la mejora continua y eliminar todos los lastres de tiempo posible en el proyecto. Esto permitía tener más tiempo para poder mejorar la calidad lo máximo posible al disponer de más tiempo para centrarnos en los detalles


