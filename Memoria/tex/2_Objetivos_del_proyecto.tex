\capitulo{2}{Objetivos del proyecto}
En este apartado apartado se van a exponer los diferentes objetivos planteados inicialmente en el proyecto divididos en varios apartados.
\subsection{Objetivos generales}
\begin{itemize}
    \item Dar a conocer los resultados de investigaciones y metodologías desarrolladas por el Didáctica de la Historia y de las Ciencias Sociales de la Universidad de Burgos.
    \item Desarrollar una página web para la consulta de libros sobre la prehistoria aptos para todos los públicos en base a criterios científicamente fundamentados sobre roles de género.
    \item Proveer herramientas de gestión de la web para las personas que administren la página.
    \item Generar herramientas de seguridad para el control y limitación de las personas colaboradoras de la web.
    \item Almacenar todos los datos en una base de datos persistente y correctamente estructurada.
    \item Importar y exportar datos de la web con facilidad para asegurar su conservación, sincronización y uso de estudios externos.

\end{itemize}

\subsection{Objetivos técnicos}
\begin{itemize}
    \item Implementar \textit{web scraping} y consultas a APIs públicas, junto a una lógica para trabajar con los datos en el \textit{backend} de la web. Todo esto se utilizaría para la búsqueda en distintas fuentes bibliográficas.
    \item Crear una calculadora web que permita estimar en base a una metodología validada el porcentaje de realidad científica existente centrado en roles de género.
    \item Implementación de diferentes roles de uso de la aplicación para limitar el acceso, de manera dinámica, a ciertas acciones de administración de la página.
    \item Utilizar un sistema de control de versiones como GitHub.
    \item Utilizar metodologías ágiles como \textit{SCRUM} y herramientas de organización de trabajo como \textit{Kanban}.
    \item Implementar y validar el interfaz web utilizando el \textit{framework} de Python Flask.
    \item Publicación del proyecto en un servicio como Netlify o Render para su acceso público.

\end{itemize}

\subsection{Objetivos personales}
\begin{itemize}
    \item Formarme en el desarrollo web.
    \item Formarme en el desarrollo de una aplicación full-stack.
    \item Brindar herramientas que ayuden a mejorar la educación.
    \item Aplicar los conocimientos adquiridos durante la carrera y las prácticas curriculares.
\end{itemize}

