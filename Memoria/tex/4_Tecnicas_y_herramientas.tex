\capitulo{4}{Técnicas y herramientas}

\section{Análisis Comparativo entre Google Books API y Amazon Books API}

\subsection{Accesibilidad y Documentación}
\begin{itemize}
    \item \textbf{Google Books API:} Ofrece accesibilidad superior y documentación detallada. Proporciona una clave de API gratuita con un límite de 1,000 solicitudes diarias.
    \item \textbf{Amazon Books API:} Requiere afiliación a Amazon Advertising API y está orientada hacia usuarios con propósitos comerciales. La documentación es robusta pero menos intuitiva.
\end{itemize}

\subsection{Amplitud de Datos Disponibles}
\begin{itemize}
    \item \textbf{Google Books API:} Acceso a más de 25 millones de libros con información extensa, ideal para proyectos educativos o bibliotecarios.
    \item \textbf{Amazon Books API:} Proporciona datos orientados a ventas y reseñas, incluyendo rankings y precios, útil para análisis de mercado.
\end{itemize}

\subsection{Facilidad de Integración y Uso}
\begin{itemize}
    \item \textbf{Google Books API:} Fácil integración gracias a su estructura basada en REST y compatibilidad con múltiples lenguajes de programación.
    \item \textbf{Amazon Books API:} Requiere comprensión avanzada de las API de Amazon y sus requisitos de autenticación, representando una curva de aprendizaje más pronunciada.
\end{itemize}

\subsection{Restricciones de Uso y Limitaciones}
\begin{itemize}
    \item \textbf{Google Books API:} Tiene limitaciones en el número de solicitudes diarias, pero generalmente suficientes para muchos proyectos.
    \item \textbf{Amazon Books API:} Limitaciones más estrictas en cuanto a la frecuencia de las solicitudes y acceso a ciertos datos.
\end{itemize}

\section{Justificación para la Elección de Google Books API}
La elección de la API de Google Books se justifica por su accesibilidad, amplia gama de datos bibliográficos, facilidad de integración, y cuota generosa de solicitudes gratuitas. Esto la hace ideal para un entorno académico o de investigación, donde los recursos pueden ser limitados y la facilidad de uso es crucial.









Esta parte de la memoria tiene como objetivo presentar las técnicas metodológicas y las herramientas de desarrollo que se han utilizado para llevar a cabo el proyecto. Si se han estudiado diferentes alternativas de metodologías, herramientas, bibliotecas se puede hacer un resumen de los aspectos más destacados de cada alternativa, incluyendo comparativas entre las distintas opciones y una justificación de las elecciones realizadas. 
No se pretende que este apartado se convierta en un capítulo de un libro dedicado a cada una de las alternativas, sino comentar los aspectos más destacados de cada opción, con un repaso somero a los fundamentos esenciales y referencias bibliográficas para que el lector pueda ampliar su conocimiento sobre el tema.