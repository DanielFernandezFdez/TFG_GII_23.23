\capitulo{4}{Técnicas y herramientas}


\section{Framework utilizado para el desarrollo del proyecto}

\subsubsection{Herramienta Elegida: Flask}

Flask es un microframework para Python que ha sido diseñado para facilitar el desarrollo de aplicaciones web. A diferencia de otros frameworks más complejos y rígidos, Flask proporciona la flexibilidad necesaria para adaptar la estructura del proyecto a las necesidades específicas de la web de libros de la prehistoria. Ofrece un conjunto de herramientas esenciales que simplifican procesos como el enrutamiento, la gestión de sesiones y la integración con bases de datos.

\subsubsection{Funcionalidades y Ventajas de Flask}

La elección de Flask se debe a su capacidad para manejar tanto aspectos básicos como avanzados del desarrollo web:

\begin{itemize}
    \item Desarrollo Ágil: Flask permite un rápido desarrollo y prototipado, lo que es ideal para los ciclos de iteración del TFG.
    \item Simplicidad y Flexibilidad: Su simplicidad  facilita la comprensión del código, lo que es fundamental para un TFG, donde la claridad y la documentación son clave.
    \item Ecosistema Completo: Existe una amplia variedad de extensiones disponibles que permiten añadir funcionalidades adicionales según sea necesario, sin sobrecargar el núcleo de la aplicación.
    \item Licencia: Flask está disponible bajo la licencia BSD, una licencia de software libre que permite la reutilización y distribución del código con pocas restricciones.
\end{itemize}
 


\section{Desarrolllo de prototipo Web}

\paragraph{Herramienta Elegida: Justinmind}

Justinmind es una herramienta de prototipado interactiva utilizada en este proyecto para diseñar la interfaz de la web de libros de la prehistoria. Fue elegida por su facilidad de uso, y amplia biblioteca de widgets.

\paragraph{Ventajas de Justinmind}

\begin{itemize}
    \item Interactividad: Simulación realista de la interfaz final de usuario
    \item Versatilidad: Adecuada para prototipos de baja y alta fidelidad.
\end{itemize}

\paragraph{Uso en el Proyecto}

Justinmind ha sido fundamental para definir y validar la experiencia del usuario, realizar pruebas de usabilidad y facilitar la comunicación del diseño entre los desarrolladores y los stakeholders.

 



\section{Análisis Comparativo entre Google Books API y Amazon Books API}

\subsection{Accesibilidad y Documentación}
\begin{itemize}
    \item \textbf{Google Books API:} Ofrece accesibilidad superior y documentación detallada. Proporciona una clave de API gratuita con un límite de 1,000 solicitudes diarias.
    \item \textbf{Amazon Books API:} Requiere afiliación a Amazon Advertising API y está orientada hacia usuarios con propósitos comerciales. La documentación es robusta pero menos intuitiva.
\end{itemize}

\subsection{Amplitud de Datos Disponibles}
\begin{itemize}
    \item \textbf{Google Books API:} Acceso a más de 25 millones de libros con información extensa, ideal para proyectos educativos o bibliotecarios.
    \item \textbf{Amazon Books API:} Proporciona datos orientados a ventas y reseñas, incluyendo rankings y precios, útil para análisis de mercado.
\end{itemize}

\subsection{Facilidad de Integración y Uso}
\begin{itemize}
    \item \textbf{Google Books API:} Fácil integración gracias a su estructura basada en REST y compatibilidad con múltiples lenguajes de programación.
    \item \textbf{Amazon Books API:} Requiere comprensión avanzada de las API de Amazon y sus requisitos de autenticación, representando una curva de aprendizaje más pronunciada.
\end{itemize}

\subsection{Restricciones de Uso y Limitaciones}
\begin{itemize}
    \item \textbf{Google Books API:} Tiene limitaciones en el número de solicitudes diarias, pero generalmente suficientes para muchos proyectos.
    \item \textbf{Amazon Books API:} Limitaciones más estrictas en cuanto a la frecuencia de las solicitudes y acceso a ciertos datos.
\end{itemize}

\section{Justificación para la Elección de Google Books API}
La elección de la API de Google Books se justifica por su accesibilidad, amplia gama de datos bibliográficos, facilidad de integración, y cuota generosa de solicitudes gratuitas. Esto la hace ideal para un entorno académico o de investigación, donde los recursos pueden ser limitados y la facilidad de uso es crucial.









