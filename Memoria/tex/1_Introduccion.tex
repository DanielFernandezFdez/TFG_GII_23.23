\capitulo{1}{Introducción}
En el contexto actual, donde la digitalización de la información juega un papel crucial en numerosos sectores, este trabajo de fin de grado introduce una aplicación web innovadora enfocada en el ámbito específico de la literatura infantil prehistórica. Esta aplicación está diseñada para facilitar el almacenamiento, la gestión y la clasificación de libros infantiles que abordan la temática prehistórica, con un enfoque particular en la representación de los roles de género conforme a los descubrimientos científicos más recientes.

La literatura infantil prehistórica ofrece una ventana única al pasado, permitiendo a los jóvenes lectores explorar cómo vivían, interactuaban y se organizaban las sociedades antiguas. Sin embargo, es crucial que estas representaciones sean fieles a los avances científicos actuales en cuanto a los roles de género, evitando perpetuar estereotipos desfasados o inexactitudes históricas.

Nuestra aplicación aborda esta necesidad al proporcionar una plataforma donde los libros infantiles sobre la prehistoria pueden ser catalogados y mostrados según su precisión y representación de los roles de género. Esto incluye una evaluación de cómo cada libro representa las dinámicas de género en contextos prehistóricos, asegurando que reflejen los conocimientos y descubrimientos científicos más recientes.

Una característica distintiva de esta herramienta es su capacidad para destacar libros que promueven una comprensión informada y actualizada de los roles de género en la prehistoria. Esto es especialmente valioso para bibliotecarios, educadores y padres que buscan ofrecer a los niños una perspectiva equilibrada y educativa, que fomente el pensamiento crítico y la comprensión de la diversidad y la igualdad de género desde una edad temprana.

Además, la aplicación no solo sirve como un repositorio de información, sino que también actúa como una guía para seleccionar material de lectura que esté alineado con los hallazgos científicos actuales, proporcionando así una base sólida para la educación y la sensibilización sobre los roles de género en diferentes épocas históricas, empezando por la prehistoria.



\textbf{FALTA ESTRUCTURA DE LA MEMORIA Y RESTO DE MATERIALES ENTREGADOS}
