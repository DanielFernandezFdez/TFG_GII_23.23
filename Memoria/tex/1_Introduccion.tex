\capitulo{1}{Introducción}
«Durante más de siglo y medio, las interpretaciones que se han hecho de los restos arqueológicos han contribuido en gran medida a invisibilizar a las mujeres prehistóricas, sobre todo al reducir su importancia en la economía», según Marylène Patou-Mathis~\cite{CitaIntro}.

En el contexto actual, donde la digitalización de la información juega un papel crucial en numerosos sectores, este trabajo de fin de grado introduce una aplicación web innovadora enfocada en el ámbito específico de la literatura infantil prehistórica. Esta aplicación está diseñada para facilitar el almacenamiento, la gestión y la clasificación de libros infantiles que abordan la temática prehistórica, con un enfoque particular en la representación de los roles de género conforme a los descubrimientos científicos más recientes.

La literatura infantil prehistórica ofrece una ventana única al pasado, permitiendo a los jóvenes lectores explorar cómo vivían, interactuaban y se organizaban las sociedades antiguas. Sin embargo, es crucial que estas representaciones sean fieles a los avances científicos actuales en cuanto a los roles de género, evitando perpetuar estereotipos desfasados o inexactitudes históricas.

Nuestra aplicación aborda esta necesidad al proporcionar una plataforma donde los libros infantiles sobre la prehistoria pueden ser catalogados y mostrados según su precisión y representación de los roles de género. Esto incluye una evaluación de cómo cada libro representa las dinámicas de género en contextos prehistóricos, asegurando que reflejen los conocimientos y descubrimientos científicos más recientes.

Una característica distintiva de esta herramienta es su capacidad para destacar libros que promueven una comprensión informada y actualizada de los roles de género en la prehistoria. Esto es especialmente valioso para el personal de bibliotecas, docentes y familias que buscan ofrecer a los niños y niñas una perspectiva adecuada y didáctica, que fomente el pensamiento crítico y la comprensión de la diversidad y la igualdad de género desde una edad temprana.

Además, la aplicación no solo sirve como un repositorio de información, sino que también actúa como una guía para seleccionar material de lectura que esté alineado con los hallazgos científicos actuales, proporcionando así una base sólida para la educación y la sensibilización sobre los roles de género en diferentes épocas históricas, empezando por la prehistoria.

\subsection{Estructura de la memoria}
Este proyecto consta de dos documentos, una memoria y un documento de anexos, los cuales se complementan para documentar completamente este proyecto.
La estructura de la memoria es la siguiente:

\textbf{Introducción}:  Descripción inicial del problema y la solución creada. Estructura de la memoria y los materiales adjuntos.

\textbf{Objetivos del proyecto}: Listado de los objetivos a cumplir durante la realización del proyecto.

\textbf{Conceptos teóricos}: Explicación de los conceptos teóricos clave implementados en la solución propuesta.

\textbf{Técnicas y herramientas}: Técnicas y herramientas utilizadas o consideradas para el desarrollo del proyecto.

\textbf{Aspectos relevantes del desarrollo}: Consideración de los aspectos destacables que tuvieron lugar durante la realización del proyecto.

\textbf{Trabajos relacionados}: Proyectos que he utilizado para poder tomar referencias y desarrollar la aplicación.

\textbf{Conclusiones y líneas de trabajo futuras}: Conclusiones obtenidas tras la finalización del proyecto y opciones de desarrollo a futuro disponibles.

Además de la estructura de la memoria, procedo a comentar la estructura de los anexos:

\textbf{Plan del proyecto}: Planificación temporal del desarrollo del proyecto y estudio de viabilidad económico y legal del proyecto.

\textbf{Requisitos}: Se contempla el listado de requisitos funcionales así como el diagrama de casos de uso explicado.

\textbf{Diseño}: Se mencionan las diferentes partes del diseño de la aplicación, como el diseño de los datos y el diseño arquitectónico.

\textbf{Manual del programador}: Incluye los aspectos más importantes del código fuente y una guía detallada de los pasos a realizar en caso de querer continuar con el desarrollo.

\textbf{Manual de usuario}: Manual de usuario para el  manejo eficiente de la aplicación.


\subsection{Materiales Adjuntos}
A continuación se muestra un listado con los materiales adjuntos a este documento:
\begin{itemize}
    \item \textit{Frontend} desarrollado en Angular, utilizado para la interfaz de usuario y se encarga de realizar las llamadas necesarias a la API.
    \item API REST en Python que contiene el \textit{backend}. Contiene toda la lógica necesaria en base a los requisitos funcionales establecidos en los anexos.
    \item Aplicación web del proyecto Prehistoria en igualdad desplegada donde el equipo de administración ha comenzado con la carga de libros y datos reales.\footnote{Acceso a la \href{https://prehistoriaenigualdad.netlify.app/}{Web desplegada}}
    \item Repositorio del proyecto que contiene todo el desarrollo del proyecto.\footnote{Acceso al \href{https://github.com/DanielFernandezFdez/TFG\_GII\_23.23}{repositorio de GitHub}}
\end{itemize}
 