\capitulo{1}{Introducción}
En el contexto actual, donde la digitalización de la información juega un papel crucial en numerosos sectores, este trabajo de fin de grado introduce una aplicación web innovadora enfocada en el ámbito específico de la literatura infantil. Esta aplicación está diseñada para facilitar el almacenamiento, la gestión y la clasificación de libros infantiles, poniendo especial énfasis en su clasificación por género literario.

La literatura infantil, rica en variedad y profundidad, ofrece un mundo de aventuras, aprendizaje y crecimiento para los jóvenes lectores. Sin embargo, la clasificación y el acceso a esta literatura pueden ser desafiantes, especialmente cuando se busca promover una amplia gama de géneros literarios. Nuestra aplicación aborda esta necesidad al proporcionar una plataforma donde se pueden catalogar y mostrar libros infantiles según varios criterios, incluyendo género literario, autor, y contenido temático.

Una característica distintiva de esta herramienta es su capacidad para clasificar libros en función de su género literario, lo que incluye categorías tradicionales como ficción, no ficción, poesía, y más. Esta funcionalidad es particularmente útil para bibliotecarios, educadores y padres que buscan fomentar una experiencia de lectura equilibrada y diversa en los niños, asegurándose de que estén expuestos a una variedad de estilos y temas literarios.

Además, la aplicación no solo sirve como un repositorio de información, sino que también actúa como una guía para seleccionar material de lectura que cumpla con ciertas métricas de calidad y diversidad literaria. Esta perspectiva ayuda a garantizar que los niños tengan acceso a una literatura que sea enriquecedora, educativa y diversa.



\textbf{FALTA ESTRUCTURA DE LA MEMORIA Y RESTO DE MATERIALES ENTREGADOS}
